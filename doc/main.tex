\documentclass[runningheads]{llncs}

\usepackage{polski} 
\usepackage[utf8x]{inputenc}
\usepackage[T1]{fontenc}
%\usepackage{makeidx}  % TODO włączyć indeks artykułu
\usepackage{graphicx}
\usepackage{hyperref}
\usepackage{algorithm}
\usepackage{algpseudocode}
\usepackage{times}

\begin{document} 
\pagestyle{headings} \addtocmark{}
%-% POCZĄTEK ARTYKUŁU
% 
\title{Wyszukiwarka postów na Twitterze}
% 
\titlerunning{Wyszukiwarka postów na Twitterze }
% 
\author{Piotr Jarosik, Michał Karpiuk, Marcin Wlazły}
% 
\authorrunning{Piotr Jarosik, Michał Karpiuk, Marcin Wlazły}
% 
\institute{Wydział Elektroniki i~Technik Informacyjnych\\
  Politechnika Warszawska\\
  \email{P.Jarosik@stud.elka.pw.edu.pl}\\
  \email{M.Karpiuk.1@stud.elka.pw.edu.pl}\\
  \email{M.Wlazly@stud.elka.pw.edu.pl}}
\maketitle
\begin{abstract}
Niniejszy praca stanowi dokumentację projektu zrealizowanego w ramach przedmiotu
\emph{Wstęp do Eksploracji Danych Tekstowych}.
Celem było zaimplementowanie prostej wyszukiwarki tekstowej. Zrealizowany system
zawiera podstawowe mechanizmy odporności na błędy w zapytaniach i treści postów
oraz uwzględnia niejednoznaczności słów. 

\keywords{wyszukiwarka postów, twitter, elasticsearch, spring, spring-data, WordNet}
\end{abstract}

W ramach projektu zrealizowane została wyszukiwarka postów Twittera 
z wykorzystaniem indeksera tekstowego opartego na silniku Lucene. Dokumentacja 
składa się z % TODO dokończyć

\section{Korpus tekstowy}
\label{sec:korpus-tekstowy}

Korpus tekstowy wykorzystany w ramach projektu składa się z dwóch zbiorów wiadomości tweeter \footnote{\url{http://an.kaist.ac.kr/~haewoon/release/twitter_tweets/}}: 
\begin{itemize}
 \item zawierających w treści wystąpienia \emph{apple};
 \item otagowane jako \verb #iranelection  .
\end{itemize}
Uwzględnione zostały wiadomości tylko w języku angielskim.

Każda wiadomość składa się z kilkunastu pól oddzielonych znakiem \verb 0x09  (\emph{ASCII horizontal tab}). W ramach
projektu uwzględnione zostały:
\begin{itemize}
  \item data utworzenia wiadomości; 
  \item nazwa użytkownika oraz jego awatar;
  \item kod kraju autora wiadomości;
  \item treść wiadomości.
\end{itemize}

Trzy pierwsze z wymienionych atrybutów przechowywane są wyłączenie w celu prezentacji wyników wyszukiwania 
użytkownikowi. Pola te nie są również uwzględniane we wstępnej analizie przed indeksowaniem. Treść wiadomości 
jest natomiast indeksowana w postaci wynikowej przygotowanej przez analizator systemu. 

\section{Wykorzystane narzędzia}
\label{sec:wykorzystane-narzedzia}

\subsection{Elasticsearch}

W ramach projektu zastosowano indekser \textbf{elasticsearch}. Działa on jako całkowicie 
niezależna usługa w systemie, nie ma więc konieczności jego uruchamiania w
serwerze aplikacyjnym  (jak jest to np. w Solr).

Podstawę działania tego narzędzia stanowi biblioteka Lucene. Instancja elasticsearch nazywana 
jest \textbf{węzłem} i może ona zawierać jeden lub wiele indeksów.
Jeden indeks elasticsearch składa się z:
\begin{itemize}
  \item jednego głównego fragmentu indeksu (ang. \emph{primary shard});
  \item jednej lub wielu replik głównego fragmentu indeksu (ang. \emph{replica shard}).
\end{itemize}
Każdy fragment indeksu elasticsearch jest \textbf{jednym indeksem Lucene}. Dzięki temu możliwa jest realizacja 
systemu niezawodnościowego - w przypadku awarii głównego fragmentu indeksu automatycznie wybierana jest jedna 
z jego replik. 

Komunikacja  między indekserem
a aplikacją odbywa się poprzez interfejs REST-owy, za pomocą  komunikatów
zapisanych w formacie JSON (stanowi on DSL \footnote{język dziedzinowy, ang. Domain Specific Language} elasticsearch'a).
W szczególności, dokumenty zapisywane  do indeksu muszą być obiektami zgodnymi z tą notacją. 

Elasticsearch posiada wiele innowacyjnych rozwiązań,  wynikających z bardzo
dużej liczby wtyczek (plugin’s) przygotowywanych przez  deweloperów z całego
świata. Szczególną ciekawą funkcją tego narzędzia są perkolacje  -  możliwe jest
wykonanie odwróconego zapytania. W indeksie można zapisać różne  frazy
wyszukiwania, a następnie przekazać do elasticsearch dokument, w  celu
znalezienia wszystkich zapisanych fraz, które  są relewantne dla wprowadzonego
dokumentu. W ten sposób można np. w prosty sposób wykonać  klasyfikację
dokumentów wprowadzanych przez użytkowników systemu.

\subsection{Spring framework}

W ramach tego projektu opracowana została aplikacja webowa z
wykorzystaniem technologii JEE. Zastosowano szkielet aplikacyjny
Spring Framework, w szczególności:
\begin{itemize}
\item \textbf{w warstwie prezentacji} -- Spring Web MVC oraz Apache
  Tiles, widoki użytkownika generowane są z plików JSP;
\item \textbf{w warstwie dostępu do danych (przechowywanych w
    indekserze)} -- narzędzie spring-data, które umożliwia
  przygotowanie abstrakcji repozytorium danych. W Spring Data nie ma
  potrzeby pisania własnych zapytań (np. obiektów w formacie JSON do
  indeksera) -- zapytania są przygotowywane automatycznie
  \textbf{wywnioskowywane} z nazw metod interfejsu
  repozytorium. Spring-data może być również zastosowany m.in. w
  relacyjnych oraz grafowych bazach danych. Istotną wadą tego
  narzędzia jest intensywne używanie refleksji języka Java -- powoduje
  to znaczny spadek wydajności systemu (w przyszłości ma być użyty
  mechanizm \verb MethodHandle  dostępny dla java 1.7).
\end{itemize}
\section{Indeksowanie danych wejściowych}
\label{sec:indeksowanie-danych-wejsciowych}
W ramach projektu wykorzystany został korpus tekstowy opisany w
rozdziale \ref{sec:korpus-tekstowy}.

Dane, przed zaindeksowaniem, zostały oczyszczone z nieistotnych
elementów (z punktu widzenia realizacji wyszukiwania), takich jak
odnośniki do zdjęć czy zewnętrznych stron (tak, aby nie były
uwzględniane w dalszej analizie). Ponadto poddane one zostały działaniu
elementów standardowego analizatora Lucene, tj:
\begin{itemize}
\item standardowy tokenizer implementujący algorytm segmentacji tekstu
  Unicode (Unicode Text Segmentation algorithm) uzupełniony o ignorowanie
  adresów url oraz email (\emph{uax\_url\_email tokenizer}). 
\item filtr \emph{stop words} dla języka angielskiego;
\item filtr zmieniający litery na małe;
\item filtr normalizujący tokeny.

\end{itemize}
Dodatkowo, zastosowane zostały następujące filtry tokenów:
\begin{itemize}
\item stemmer dla języka angielskiego;
\item filtr do przygotowywania bigramów z danych wejściowych (shingle
  filter z biblioteki Lucene);
\item filtr uwzględniający synonimy wyrazów.
\end{itemize}

Aby skorzystać z tego ostatniego, konieczne było przygotowanie pliku z
synonimami w formacie akceptowalnym przez Solr. Niezbędne było więc
skorzystanie ze słownika języka angielskiego.

W ramach projektu wykorzystane zostały dane z Wordnet-u
\footnote{\url{http://wordnet.princeton.edu/}}, który jest leksykalną
bazą danych języka Angielskiego. Z punktu widzenia projektu
szczególnie interesujące były:
\begin{itemize}
\item synonimy (niezbędne dla wskazanego wcześniej filtru analizatora);
\item dane do sprawdzania treści zapytań.
\end{itemize}

Elasticsearch umożliwia zastosowanie informacji bezpośrednio  
z Wordnet-u, jednak w związku z błędem występującym w indekserze (nie wszystkie postacie są 
obecnie akceptowane przez elasticsearch) dane te zostały skonwertowane do postaci zgodnej ze składnią
Solr.

\section{Wyszukiwanie informacji w indeksach}
\label{sec:wyszukiwanie-informacji-w-indeksach}
Zarówno indeksowane dokumenty, jak i zapytania zostały poddane działaniu tych
samych analizatorów Lucene.
Do obsługi błędów w zapytaniach wykorzystano suggester termów \footnote{nowa
funkcjonalność elasticsearch’a, ciągle jeszcze w fazie beta:
\url{http://www.elasticsearch.org/guide/reference/api/search/suggest/}}, który
na podstawie podanej frazy oraz na podstawie zaindeksowanych postów podaje
sugestie dla prawdopodobnie błędnych wyrazów, tj. słowa z dokumentów
przechowywanych w indeksie najbliższe do podanego ciągu wyrazów. Np. dla frazy
“spatła kedytu” użytkownik otrzyma informację, iż prawdopodobnie chodziło mu o
frazę “spłata kredytu”.

Suggester termów wykorzystany został również przed zaindeksowaniem postów (w
sytuacjach gdy było to konieczne).
Dla każdego wyrazu, który jest z prawdopodobnie błędny, uwzględniona jest
propozycja od suggester’a. To, czy dane słowo zostało uznane za błędne,
wyznacza wartość progowa odległości słowa z zapytania do  proponowanego przez
suggester wyrazu. Parametr ten został ustalony  w trakcie eksperymentowania z
wyszukiwaniem zaindeksowanych danych i wynosi $0.5$. 

%TODO zrobiliśmy to poniżej?
%Docelowo chcemy wyeliminować typowe błędy (literówki)  jeszcze przed
% zaindeksowaniem oryginalnego tekstu.

Użytkownik końcowy ma możliwość wpisania frazy do wyszukiwania w
specjalnie przygotowanym do tego formularzu. W odpowiedzi wyświetlana jest lista
pasujących do zapytania postów. Wyniki wyszukiwania jest stronicowana, tj.
na jednej stronie będzie można wyświetlić pewną określoną,  maksymalną liczbę
wyników. Do stronicowania została wykorzystana technologia AJAX, dzięki czemu
posty wczytują się dynamicznie, a użytkownik nie musi za każdym razem
przeładowywać strony. Każdy wynik wyszukiwania zawiera:
treść postu, awatar oraz nazwę autora. Dane są wyświetlane rosnąco wg wagi 
\verb tf-idf  .

\section{Metody testowania}
\label{sec:metody-testowania}

W ramach projektu zaindeksowano 85252 wiadomości, zarówno z korpusu tekstowego
(przedstawionego w \ref{sec:korpus-tekstowy}) jak i portalu \url{twitter.com}. 
Na danych testowych przeprowadzone zostały testy jakości wyszukiwania. 
Wykonane testy zostały zaprezentowane w kolejnych podrozdziałach:
\subsection{Testy jednostkowe}
Testy jednostkowe zostały przeprowadzone z wykorzystaniem framework’a TestNG.
Jest to ciekawy szkielet aplikacyjny stanowiący alternatywę dla biblioteki
JUnit. Umożliwia on elastyczną konfigurację testów, wykorzystuje się go przy
pomocy adnotacji Java.
 
\subsection{Testy jakościowe}
%  Testy jakościowe - wybrane posty zostały wstępnie ręcznie przeczytane i 
%   zapisane zostały wystąpienia słów oraz synonimów zawartych w testowych
%   postach.
% 	Sprawdzenie skuteczności polegało na użyciu wyszukiwarki na  tych testowych
% 	postach i porównaniu wyników wyszukiwania z wynikami powstałymi  na skutek
% 	ręcznego przeglądania postów. W wyniku tych testów zostały policzone  metryki
% 	precision oraz recall. Policzone ze wzorów:
% \begin{equation}
% 	precision = \frac{TA}{TA + FA}
% \end{equation}
% \begin{equation}
% 	recall = \frac{TA}{TA + FB}
% \end{equation}
% Gdzie: 

% 	$TA$: posty poprawnie zakwalfikowane do klasy $A$
	
% 	$FA$: posty niepoprawnie zakwalfikowane do klasy $A$
	
% 	$TB$: posty poprawnie zakwalfikowane do klasy $B$
	
% 	$FB$: posty niepoprawnie zakwalfikowane do klasy $B$
	
% Precyzja (\textit{ang.} Precision)  określa jaka część obiektów
% zaklasyfikowanych pozytywnie, jest taka  w rzeczywistości. Innymi słowy jest to
% prawdopodobieństwo tego, że obiekt faktycznie jest  pozytywny, jeśli został
% uznany jako taki przez system.
% Zupełność, kompletność (\textit{ang.} Recall) Jest to miara odpowiadająca na
% pytanie, jaką część faktycznie pozytywnych obiektów, system zaklasyfikował poprawnie.

W ostateczności nie zdecydowaliśmy się na próby  obliczania miar poprawności
takich jak precyzja czy zupełność. Byłoby to  wyjątkowo trudnym zadaniem, biorąc
pod uwagę fakt, że wyniki testów w bardzo dużym  stopniu zależałyby od
skomplikowania zapytania, co również jest kwestią  subiektywną. Takie wyniki
były by raczej mało precyzyjne. Ponadto aby obliczyć  wskaźnik zupełności
niezbędna jest wiedza o wszystkich poprawnych przypadkach  postów w korpusie, w
kontekście danego zapytania. Taką wiedzę można zdobyć  uważnie przeglądając
wszystkie posty, jednak nakład pracy potrzebny na jej  zdobycie jest
zdecydowanie nieakceptowalny. Z drugiej strony  ograniczenie zbioru testowego do
wielkości umożliwiającej zapanowanie nad testami bardzo zmniejszyłoby
dokładność.

W związku z tym przeprowadzone zostały testy dla wybranych zapytań.

\newpage
\textbf{Testy poprawności wyszukiwania}\\

\begin{enumerate}
\item Wyszukiwanie proste, fraza wejściowa: \emph{apple juice}.\\
   Liczba wszystkich wyników: 2907. 
   Najbardziej relewantny wynik (tf-idf =  3.0314815):  
   \emph{Strawberry Kiwi Apple juice and Campbell's chunky soup for breakfast!  Yummy!}

\item Uwzględnienie synonimów, fraza wejściowa: \emph{apple succus} (bardzo rzadko występujący synonim \emph{juice}).\\
  Liczba wszystkich wyników: 2012.
  Najbardziej relewantny wynik (tf-idf = 2.4052076):
  \emph{Strawberry Kiwi Apple juice and Campbell's chunky soup for breakfast!  Yummy!}

\item Uwzględnienie synonimów oraz stemming, fraza wejściowa: \emph{apples headphone} (bardzo rzadko występujący synonim \emph{juice}).\\
  Liczba wszystkich wyników: 2012.
  Najbardziej relewantny wynik (tf-idf = 2.4052076):
  \emph{Strawberry Kiwi Apple juice and Campbell's chunky soup for breakfast!  Yummy!}

\end{enumerate}
   

\subsection{Testy wydajnościowe}
Testy wydajnościowe miały zostać przeprowadzone  za pomocą frameworka The
Grinder, który 
umożliwia łatwe i szybkie  tworzenie  testów wydajnościowych, w tym również
na oszacowanie wydajności systemu w przypadku  jego użycia przez wielu
użytkowników jednocześnie. Ze względu na brak miarodajnego środowiska
deweloperskiego (odpowiadającego środowisku produkcyjnemu) nie udało się
przeprowadzić testów wydajnościowych.

\clearpage
\bibliographystyle{plain} \bibliography{bibliografia}
\end{document}
